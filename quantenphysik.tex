

\chapter{Definitionen}


\section{Quantenphysik ?!}

Die \emph{Quantenphysik}\index{Quantenphysik} beschreibt das Verhalten von Teilchen auf atomarer Ebene und tiefer. Eine besondere Eigenschaft ist dabei, dass bestimmte Größen \emph{gequantelt}\index{gequantelt} vorkommen -- also nur bestimmte, diekrete Größen haben können. Das \textsc{Planck}'sche Wirkungsquantum $h$\index{Plakck'sches Wirkungsquantum}\index{Wirkungsquantum} ist eine bedeutende Konstante in der Quantenphysik.

In der Quantenphysik können über bestimmte Eigenschaften der beobachteten Objekte keine Aussagen gemacht werden. Bspw. kann man bei einem Quantenobjekt nicht sagen, wie es einen bestimmten Weg zurücklegt. Außerdem lassen sich manche Größen nicht beliebig genau messen. Bei Impuls \(p\) und Ort \(x\) eines Teilchens beispielsweise hängt die Messgenauigkeit der einen Größe von der Genauigkeit bei der zweiten Größe ab.

Als mathematischen Formalismus, mit dem man das Verhalten der Teilchen recht genau vorhersagen kann, dient die Zuordnung einer \emph{Wahrscheinlichkeitswelle} $\Psi(x)$\index{Wahrscheinlichkeitswelle} zu jedem bestimmten Teilchen. Ihr Quadrat $\Psi^2$\index{Psi ($\Psi$)} ist die Wahrscheinlichkeit, das Teilchen am betreffenden Punkt anzutreffen.

Insgesammt richtet es sich in der Quantenphysik oft nach Wahrscheinlichkeiten....


\section{Wesenszüge der Quantenmechanik}

\begin{description}
   \item[Statistisches Verhalten] Man kann für ein einzelnes Teilchen nicht ausrechnen, wo es auftreffen wird, sondern nur eine Wahrscheinlichkeit dafür. Wiederholt man einen Versuch oft genug, wo werden viele Teilchen sich an diese Wahrscheinlichkeitsverteilung halten.\index{Wahrscheinlichkeitsverteilung} Abgesehen von stochastischen Abweichungen ist dann die Verteilung der angetroffenen Teilchen gleich der berechneten Aufenthaltswahrscheinlichkeit.
   \item[Interferenzfähigkeit] Einzelne Quantenobjekti können mit sich selbst interferieren. Voraussetzung dafür ist, dass es für den Weg mehrere (klassische) Möglichkeiten gibt. Auf diesem Weg interferieren nun die Wahrscheinlichkeitswellen miteinander und erzeugen die Auftreffwarhscheinlichkeit.\index{Interferenz!Wahrscheinlichkeitswellen}\index{Interferenz!Quanten}
   \item[Objektiv unbestimmbar] Bei mehreren (klassischen) Wegmöglichkeiten zur realisierung eines Zustands ist es nicht möglich, zu sagen, welchen das Quantenteilchen auch wirklich eingeschlagen hat. Vielmehr hat es schlichtweg \emph{keinen} der möglichen Wege realisiert. Im Bezug auf manche Größen ist es nicht möglich sie einem Quantenteilchen zuzuordnen.
   \item[Eindeutige Messung] Wenn mehrere (klassische) Möglichkeiten vorliegen, ist es möglich, zu messen, welche dieser Möglichkeiten das Quantenobjekt realisiert, indem man es durch die Messung in den klassischen Zustand drängt. Man misst also nicht mehr die Eigenschaft des Quantenobjekts, weil dieses ja keinen der zu messenden Zustände realisiert hat.
   \item[Komplementarität]\index{Komplementarität} \emph{Wegen} dieser Messung wird das Quantenteilchen sich von nun an aber wie ein klassisches Teilchen verhalten. Es wurde gewissermaßen "`gezwungen"', einen klassischen Zustand zu realisieren und verbleibt dann in diesem. Man kann also entweder messen, wo sich ein Teilchen aufhält oder seine Quanteneigenschaften -- wie bspw. Interferenz -- beobachten.
 \end{description}



\chapter{Interferenz}



	\section{Messung unbestimmter Methoden}


\subsubsection{Messungen auf beiden möglichen Wegen}
Wenn auch ein Wesenszug der Quantenmechanik besagt, dass das Teilchen sich an einem unbestimmten Ort aufhält, so kann man diesen genau bestimmen -- so ein weiterer Wesenszug. Misst man, wo sich ein Quantenteilchen aufhält, so bekommt man dafür eine genaue Information. Diese Information kann nun dazu führen, dass Interferenz (bspw. am Doppelspalt) unterbunden wird.

Schickt man Quantenteilchen auf einen Einzelspalt, so findet dort Beugung am Einzelspalt statt.\footnote{Die effektive Spaltbreite -- auf die man durch Untersuchung des Interferenzbildes schließe -- ist jedoch auch abhängig von der Wellenlänge: Nur wenn die max. Auflösung den kompletten Spalt abdeckt, findet die INterferenz am Einfachspalt ungestört statt.}\index{Beugung!Quanten} Schickt man die Quantenobjekte dagegen -- unbeobachtet -- auf einen Doppelspalt, so ergibt sich Interferenz am Doppelspalt. Interessant ist dabei, dass die Elektronen weder den Weg durch den einen noch durch den anderen Spalt wählen und es trotzdem zu Interferenz zwsischen den $\Psi$-Wellen der beiden klassisch möglichen Wege kommt.

Setzt man nun jedoch Messinstrumente ein, mit denen man \emph{theoretisch}\footnote{Die Einschränkung kommt daher, dass man wie beim "`Quantenradierer"' in jeden Spalt des Doppelspalts je eine Polarisationsfolie einbringt, wobei die beiden Polarisationswinkel sich um $90^o$ unterscheiden. Es ist danach das Interferenzbild von der Beugung am Einzelspalt zu beobachten, weil man durch bestimmte Geräte unterscheiden könnte, durch welchen Spalt die Quantenteilchen gelangt sind, auch wenn man bei normaler Beobachtung nicht darauf schließen kann (weil wir ja kein unterschiedlich polarisiertes Licht unterscheiden können).} unterscheiden kann, durch welchen der beiden Spalten das Quantenobjekt gegangen ist, zwingt man es gewissermaßen dazu, sich zu entscheiden, durch welchen der beiden Spalte es gehen möchte. Ohne die Messinstrumente kann es eine Art "`Zaubertrick"' rund um und durch den Doppelspalt aufführen, dessen Effekt die Interferenz ist. Nun zwingt man es dagegen, sich zu entscheiden, welchen Spalt es nehmen möchte. Damit ist das Interferenzbild des Doppelspaltes ausgeschlossen, weil dazu \emph{beide} Spalte einbezogen werden müssen -- es muss \emph{Irgendetwas} aus jedem der beiden Spalte zu der Beugung beitragen. Wird das Quantenteilchen dagegen gezwungen, einen der beiden Wege zu nehmen, kann es nicht gleichzeitig noch den zweiten Weg mitbenutzen -- was ja nötig gewesen wäre, um Interferenz am Doppelspalt zu erhalten.




	\subsubsection{Messung auf einem möglichen Weg}

Seltsam wird die Sache allerdings dadurch, dass der oben beschriebene Effekt auch dann eintritt, wenn nur einer der beiden Wege überwacht wird. Geht man wieder von Interferenz am Doppelspalt aus und überwacht nur \emph{einen der beiden} Spalten auf passierende Elektronen, so wird sich auf dem Schirm trotzdem das Interferenzbild von der Beugung am Einzelspalt zeigen.

Bei zwei Wegen -- A und B -- wird nur der Weg A überwacht. \emph{Trotzdem} wird ein Teilchen gezwungen, sich zwischen den beiden Wege zu entscheiden. Nimmt ein Teilchen den Weg B, so kann es eigentlich nicht wissen, dass Weg A überwacht wird. Trotzdem lässt es sich von der Überwachung auf Weg A in das "`klassische Schema"' drücken und verliert dadurch seine Fähigkeit, den "`Zaubertrick der Interferenz"' zu vollführen.






	\subsubsection{Messung beider Wege die danach zurückgenommen wird}

Beim oben angesprochenen "`Quantenradierer"'\index{Quantenradierer} kann man nun eine weitere Polarisationsfolie einbringen. Sie wird zwischen (präpariertem) Doppelspalt und Schrim eingebracht und ihr Polarisationswinkel steht im Winkel von $44^o$ auf den Winkeln der beiden Folien in den Spalten.

\index{Polarisation}Licht, das von den Folien in den Spalten polarisiert wurde, muss nun, um auf den Schirm zu gelangen, auch diesen Filter passieren. Dabei kommen von dem Licht, das vorher in zwei senkrecht zueinander stehenden Ebenen polarisiert war, nur diejenigen Anteile durch, die im Winkel von $45^o$ zu den beiden Ebenen liegen. Nach diesem dritten Polarisationsfilter ist also alles licht wieder in die selbe Richtung (in der selben Ebene) polarisiert.

Das Licht aus den beiden Spalten hat darüber hinaus die selbe Intensität -- durch die beiden Spalte fällt die selbe Lichtmenge. Nach dem dritten Polarisationsfilter ist also nicht mehr entscheidbar, welchen der beiden Spalte das Licht passiert hat und somit ergibt sich wieder Interferenz am Doppelspalt mit dem dazugehörigen Schirmbild. Die Quantenteilchen müssen sich also am Doppelspalt nicht mehr entscheiden, weil sie "`wussten"', dass \emph{hinter} dem Spalt ein weiterer Filter kommt, durch den nicht mehr entscheidbar ist, welchen Weg sie wählten -- und dadurch mussten sie nicht mehr wählen. 


\subsubsection{Das Verwirrende also:} Die Quantenteilchen können somit also gewissermaßen in die Zukunft blicken, bzw. den Weg sehen, den sie \emph{nicht} genommen haben.










\section{Abhängigkeit von Wellenlänge des Lichts, mit dem beobachtet wird}

Interferenz tritt nur dann auf, wenn das Teilchen unbeobachtet ist -- sich also gewissermaßen ungestört als Quantenobjekt aufführen kann. So lange nimmt es keinen besonderen Weg und seine $\Psi$-Wellen können ungestört interferieren. Will man jedoch unterscheiden, welchen Weg das Teilchen nimmt, so kann es nicht mehr als Quantenteilchen "`handeln"' und wird gezwungenermaßen zum klassisch\emph{eren} Teilchen.

Damit es sich wie ein vollständig klassisches Teilchen verhält, müsste man unendlich genau beobachten, welchen Weg es nimmt. Da die Messung aber mit Licht funktioniert, ist der Genauigkeit der Ortsbestimmung durch die optische Auflösung Grenzen gesetzt. Mit Licht der Wellenlänge $\lambda$ kann man nämlich nur entscheiden, dass sich das Quantenteilchen im Bereich $\frac{1}{2} \lambda$ irgendwo aufgehalten hat.

% Beleuchtet man so also einen Doppelspalt, um herauszubekommen, aus welchem der beiden Spalte bspw. ein Elektron geflogen kommt und verwendet dazu Licht einer Wellenlänge die kleiner als der Abstand zwischen den beiden \emph{äußeren} Spaltenden ist ("`Spaltaußenabstand"'), so wird man auf dem Schirm stets ein Interferenzbild erhalten, das nach Interferenz am \emph{Einzelspalt} aussieht. Die Elektronen können nämlich nicht aufgelöst werden, aus welchem Spalt sie kommen und somit kann sich ungestört Interferenz am Doppelspalt für jedes einzelne Teilchen ergeben.
% 
% Senkt man nun jedoch die Wellenlänge des Lichts ab, sodass sie kleiner ist als der Spaltaußenabstand, so kann man mehr und mehr unterscheiden, aus welchem Spalt ein Elektron kommt. Ist die Wellenlänge nur wenig kleiner als der Spaltaußenabstand, so 

\index{Interferenz!Quanten}Es ergibt sich der Zusammenhang (Daten siehe Tab. \ref{tab_daten_lambda_0}) zwischen Spaltabstand $g$, Spaltbreite $l$ und Wellenlänge $\lambda_0$, unterhalb der sich nurnoch Interferenz am Einzelspalt ergibt:
\begin{equation}
   2 \cdot g - 2 \cdot l = 2 \cdot (g - l) = \lambda_0
\end{equation}
Damit entspricht die Wellenlänge $\lambda_0$ dem Doppelten des "`Balkens"' zwischen den beiden Spalten des Doppelspalts. Oberhalb dieser Wellenlänge erhält man teilweise das Interferenzbild am Doppelspalt. Das volle Bild des Doppelspalts\footnote{Definitionsgemäß wurde das angenommen, sobald die totalen Minima um das erste Hauptmaxima die Intensität $0$ hatten.} erhält man dagegen (Daten siehe Tab. \ref{tab_daten_lambda_1}) erst mit Licht der Wellenlänge $\lambda_1$:
\begin{equation}
   2 \cdot g = \lambda_1
\end{equation}
Erklären kann man sich diese Zusammenhänge dadurch, dass die Auflösung einer Lichtwelle durch deren Wellenlänge festgelegt ist -- und zwar ist die maximale Auflösungslänge gleich der Hälfte der Wellenlänge. Das würde auch zur \textsc{Abbe}'schen Abbildungsbedingung (für Lichtmikroskope) passen. Sie sagt, die Max. Auflösung $d = \frac{\lambda}{2 \cdot n \cdot sin(\alpha)}$. Da wir kein Mikroskop mit beschränkung haben, ergibt sich für unsere Beobachtung der Winkel $\alpha = 90^o$, da wir in Luft arbeiten ergibt sich $n \approx 1$ und somit $d = \frac{\lambda}{2}$.\index{Auflösung}\index{Abbe'sche Abbildungsbedingungen}

\index{Beobachtung von Quanten}Sobald nun die Auflösung kleiner ist als der Steg zwischen den beiden Spalten, ist es auf keinen Fall mehr möglich, beide Spalte mit dem selben Lichtblitz zu beobachten. Wäre die Wellenlänge ein klein bisschen größer, könnte es im ungünstigsten Falle sein, dass die Lichtwelle genau auf die Spaltmitte fällt und somit von beiden Spalten einen Teil betrachtet -- nun könnte ein Elektron also weiterhin durch beide Spalte gleichzeitig gelangen und würde trotzdem von der selben Welle registriert. Erst wenn $\lambda < \lambda_0$ gilt, ist dies ausgeschlossen und es kann immer nur ein Spalt gleichzeitig beobachtet werden. Somit ist absolut klar, aus welchem Spalt das Elektron geflogen kam und somit ist Interferenz zwischen einem Elektron mit sich selbst aus zwei verschiedenen Spalten völlig ausgeschlossen.

Bei größerer Wellenlänge ($\lambda_0 < \lambda < \lambda_1$) ist es nun für Elektronen möglich, immernoch beide Spalte zu passieren und trotzdem von dem selben Lichtblitz wahrgenommen zu werden. Ist $\lambda$ nur geringfügig größer als $\lambda_0$, so ist es wenig wahrscheinlich, dass ein Elektron dieses Kunststück zu Werke bringt. Hier überlagern sich die beiden Fälle von Interferenz: Einmal am Doppelspalt von sehr wenigen Elektronen und einmal am Einzelspalt von sehr vielen Elektronen. Das Bild sieht deshalb noch weitgehend aus wie bei Beugung am Einzelspalt. Je mehr sich $\lambda$ dann $\lambda_1$ annährt, desto mehr Elektronen ist es möglich, das Kunststück zu vollführen.

Wenn dann $\lambda > \lambda_1$ ist ein neuer Fall der totalen Unkontrollierbarkeit eingetreten: Es ist jetzt überhaupt nicht mehr möglich, zu entscheiden, aus welchem Spalt ein bestimmtes Elektron gekommen ist -- und folglich entsteht ab jetzt ein Interferenzbild des Doppelspalts auf dem Schirm. 

~\\Was man die ganze Zeit über beachten muss ist, dass $\lambda < l$ gelten müsste, damit die Interferenz am Einfachspalt ebenfalls eingeschränkt wird. Es ergibt sich daraus eine neue Grenzwellenlänge:
\begin{equation}
   2 \cdot l = \lambda_2
\end{equation}
Unterhalb dieser Wellenlänge ergibt sich nurnoch Interferenz am Einzelspalt, außerdem sieht das interferenzbild so aus, als ob der Spalt nur $l' = \frac{\lambda}{2}$ breit ist.




\begin{table}
\centering
\subtable[
	Bestimmung von $\lambda_0$
]{
	\begin{tabular}{c c c}
	%Spaltabstand [nm] & Spaltbreite [nm] & vollst. Einzelspaltinterferenz unterhalb Wellenlänge [nm]\\
	$g$ [nm]& $l$ [nm]& $\lambda_0$[nm]\\
	400	&	100	&	600\\
	400	&	150	&	500\\
	400	&	200	&	400\\
	300	&	100	&	400\\
	350	&	100	&	500\\
	400	&	100	&	600\\
	450	&	100	&	700
	\end{tabular}\label{tab_daten_lambda_0}
}
\subtable[
	Bestimmung von $\lambda_1$
]{
	\begin{tabular}{c c c}
		%Spaltabstand [nm] & Spaltbreite [nm] & vollst. Doppelspaltinterferenz oberhalb Wellenlänge [nm]\\
		$g$ [nm]& $l$ [nm]& $\lambda_1$[nm]\\
		
		250 & 100 & 500\\
		275 & 100 & 550\\
		300 & 100 & 600\\
		325 & 100 & 650\\
		350 & 100 & 700\\
		250 & 100 & 500\\
		200 & 200 & 400\\
		250 & 125 & 500\\
		250 & 150 & 500\\
		300 & 100 & 600\\
		300 & 125 & 600\\
		300 & 150 & 600\\
		300 & 200 & 600
	\end{tabular}\label{tab_daten_lambda_1}
}
\caption{Ergebnisse aus Versuchen mit Elektronen $\lambda_{Elektron} = 4pm$}
\label{tab_versuch_zshg_spalten_wellenlaenge}

\end{table}




















\chapter{Mathematischer Formalismus}

Um das Verhalten von Quantenteilchen vorherzusagen, ordnet man ihnen eine $\Psi$-Welle\index{Psi ($\Psi$)}\index{Psi ($\Psi$)!Psi-Welle} zu. Dazu ordnet man jedem Teilchen, abhängig von Seiner Energie $W$, eine Wellenlänge zu. Um die Welle aufzustellen bekommt deswegen auch ein Elektron eine Wellenlänge. Diese "`\textsc{DeBroglie}-Wellenlänge"'\index{DeBroglie-Wellenlänge} berechnet sich nach
\begin{equation}
   \lambda = \frac{h}{m \cdot v} = \frac{h}{p}
   \label{eq_debroglie}
\end{equation}
Um die Welle mathematisch auszudrücken benötigt man jedoch \emph{Komplexe Zahlen}.\index{Komplexe Zahlen} Man kann sich für die Funktion einen rotierenden Zeiger denken, dessen $y$-Komponente imaginär und dessen $x$-Komponente reell ist. Da die Amplitude des Zeigers jedoch konstant ist, ist auch die Aufenthaltswahrscheinlichkeit $\Psi^2$ eines Teilchens entlang einem rotierenden "`Zeigerstrahl"' konstannt.
Für $\Psi(x)$ ergibt sich:
\begin{equation}
   \Psi(x) = \Psi_0 \cdot cos(\varphi_x) + \Psi_0 \cdot i \cdot sin(\varphi_x) = \Psi_0 \cdot cos \left( \frac{2 \pi \cdot x}{\lambda} \right) + \Psi_0 \cdot i \cdot sin \left( \frac{2 \pi \cdot x}{\lambda} \right)
   \label{eq_psi(x)}
\end{equation}
Nach der Zeit $t$ müssen alle Zeiger um den Betrag $\varphi_t = \frac{2\pi \cdot t}{T}$ "`zurückgedreht"' werden, um wieder auf die durch Formel \ref{eq_psi(x)} gebracht zu werden. Dadurch ergibt sich eine Phase für jeden Zeiger von $\varphi = \varphi_x - \varphi_t$ und damit für die Wellengleichung:
\begin{equation}
   \Psi(x,t) = \Psi_0 \cdot  cos \left(\frac{2\pi\cdot x}{\lambda} - \frac{2\pi \cdot t}{T}\right) + \Psi_0 \cdot i \cdot sin \left(\frac{2\pi\cdot x}{\lambda} - \frac{2\pi \cdot t}{T} \right)
\end{equation}
\index{Wellengleichung!Quanten}Über diese Wellengleichungen kann man nun Interferenzen der Teilchen berechnen. Interessant ist dabei, dass man die Interferenzen von einzelnen Teilchen \emph{mit sich selbst} berechnet. Um ein messbares Ergebnis zu erhalten addiert man die $\Psi$-Wellen zur Zeit $t_0$ am Ort $x_0$ wie gewohnt um dann die Auftreffwahrscheinlichkeit eines einzelnen Teilchens mit $\Psi_{sum}^2(x_0,t_0)$ zu berechnen.






\chapter{Abweichungen von Klassischen Vorstellungen}



\section{Photoeffekt}


\subsubsection{Versuchsdurchführung}

%\subsubsection{Photoeffekt}

\index{Photoeffekt}Eine Metallplatte wird mit monochromatischem Licht bestrahlt. Vor der Metallplatte sitzt ein leitender Ring, durch den das Licht einfällt. Zwischen dem Ring und der Metallplatte wird eine Spannung $U_A$ angelegt und die Stromstärke gemessen. Die Spannung wird während des Versuchs für jede Frequenz so lange erhöht, bis gerade keine Stromstärke mehr zu messen ist. Dann ist die Spannung $U_A$ gerade so groß, dass sie die losgelösten Elektronen gerade so vom Ring abhält\footnote{Während noch Strom floss, schafften es immernoch Elektronen, sich bis zum Ring gegen das E-Feld zu bewegen. Das Elektron der Elementarladung $e$ benötigt für die Strecke $s$ die Energie $W = E \cdot e \cdot s$}. Jetzt gilt also \begin{equation}
U_A \cdot e = E_{kin} = h \cdot f - W_A 
\end{equation}
Ist $W_A$ für das Element bekannt, kann man nun das \textsc{Planck}'sche Wirkungsquantum $h$ bestimmen.


	\subsubsection{Erkenntnisse}
	
Beim \emph{Photoeffekt} wird eine Metallplatte mit Licht bestrahlt. Dieses Licht wechselwirkt mit Elektronen auf der Platte. Ist das Licht energiereich genug, so kann es die Elektronen von der Platte lösen, dafür ist pro ELektron die Materialkonstante $W_A$ nötig. Alle Energie des Lichtes, die über die \emph{Ablösearbeit} $W_A$\index{Ablösearbeit} hinausgeht, wird dem Elektron als kinetische Energie $W_{kin}$ übergeben.

Überraschenderweise ergibt sich bei dem Versuch, dass die kinetische Energie $W_{kin}$ der Elektronen nur von der Frequenz $f$ des Lichts abhängt -- nicht jedoch von der \emph{Intensität} $I$\index{Intensität von Licht}. Nach klassischen Verständnis sollte die vom Licht übertragene Energie von der Amplitude der Elektromagnetischen Welle, damit $\vec{E}$ und damit $I$ ($I \sim \vec{E}^2$) abhängen. Die Intensität beeinflusst jedoch nur wie \emph{viele} Elektronen herausgelöst werden; nicht jedoch deren Energie. Für Lichtwellen gilt der Zusammenhang zwischen Frequent $f$ und enthaltener Energie $W$:
\begin{equation}
   W = h \cdot f
   \label{eq_W=hf}
\end{equation}
Licht einer einheitlichen Frequenz $f$, das von einer Lampe ausgestrahlt wird, überträgt also \emph{diskrete}\index{diskret} Energieportionen der Größe $n \cdot h \cdot f$ an andere Teilchen. Eine solche Energieportion wird als \emph{Photon} bezeichnet. Die \emph{Intensität} des Lichts bestimmt lediglich, wie viele Photonen ausgesandt werden.\index{gequantelt}
Für den Photoeffekt ergibt sich so für die Kinetische Energie $W_{kin}$:
\begin{equation}
   W_{kin} = h \cdot f - W_A
\end{equation}




% \subsubsection{Frank Hertz Verusch}
% 
% Beim Frank-Hertz-Versuch werden statt Photonen Elektronen verwendet. Sie werden durch eine Beschleunigungsspannung $U_B$ auf eine bestimmte Energie $W = U_B \cdot e$ gebracht. Danach treten sie in ein bremsendes E-Feld der Stärke $E$ ein. Hier befinden sich ebenfalls Gasatome.




		\section{Unbestimmtheitsrelation}


In der Quantenphysik gibt es bestimmte Paare an Größen, deren maximal mögliche Messgenauigkeiten voneinander abhänngen. Die bedeutendsten davon sind Ord $x$ und Impuls $p$. Dabei ist es unabhängig vom Versuchsaufbau, der Durchführung etc. nicht möglich, beide größen beliebig genau zu messen. Der Zusammenhang lässt sich bei Beugung am einzelnen Spalt leicht herleiten. \index{Unschärferelation}\index{Unbestimmtheitsrelation}

Dadurch, dass man die Spaltbreite $l$ vorgibt, kann man sich sicher sein, dass alle Quantenteilchen, die später auf dem Schirm zu sehen sind, den Spalt passiert haben. Im Bereich des Spaltes kann man also den Ort des Quantenteilchens bis auf die Strecke $l$ genau bestimmen. Somit gilt 
\begin{equation}
   \Delta x = l
   \label{eq_ubr_x}
\end{equation}
Für den Impuls $p_x$ parallel zur Spaltebene ergibt sich für ein Teilchen, das um den Winkel $\varphi$ abgelenkt wurde, sodass es (beim Schirm) konstruktiv mit sich interferiert
\begin{equation}
   \frac{p_x}{p} = sin(\varphi)
   \label{eq_ubr_px}
\end{equation}
%Über den Zusammenhang \ref{eq_debroglie} aufgelöst nach $p$ und eingesetzt in \ref{eq_ubr_px} ergibt sich
Für das Teilchen gilt nach der \textsc{Fraunhofer}-Näherung für konstruktive Interferenz
\begin{equation}
l \cdot sin(\varphi) = \lambda   
\label{eq_konstr_interf}
\end{equation}
Nun ersetzt man in \ref{eq_debroglie} das $p$ durch das aus \ref{eq_ubr_px}, löst dies nach $\lambda$ auf und setzt diese in \ref{eq_konstr_interf} ein. somit ergibt sich
\begin{equation}
   \Delta x \cdot \Delta p_x = h
   \label{eq_ubr}
\end{equation}
Diese Gleichung (\ref{eq_ubr}) drückt die \emph{Unbestimmtheitsrelation} der beiden Größen $x$ und $p$ aus: Je genauer man $x$ messen will, desto ungenauer geht das nur für $p$.





	\section{Lokalisationsenergie}

\index{Lokalisationsenergie}Aus Gleichung \ref{eq_ubr} ergibt sich indirekt, dass ein Teilchen, das man in seiner Bewegungsfreiheit -- also dem Ort $x$ auf den Bereich $\Delta x$ einschränkt -- ein wesentlich stärker unbestimmter Impuls anhaftet. Da ein Quantenteilchen mit einer gewissen Wahrscheinlichkeit all seine Möglichkeiten ausprobiert, ist es also möglich, dass es jeden Impuls innerhalb von $\Delta p = \frac{h}{\Delta x}$ annimmt.

Nun kann man versuchen, die Energie zu ermitteln, die mit einem Impuls -- und damit auch mit der örtlichen Beschränkung -- verknüpft ist. Da $\Delta p$ ein Bereich an möglichen Impulsen darstellt, ist es dem Quantenteilchen maximal möglich den Impuls $p_x$ bzw. $-p_x$ zu erreichen, und deshalb definiert man $\Delta p = 2p_x$.

Die \emph{Lokalisationsenergie} bestimmt man nun über die Kinetische Energie $W_{kin}$, die ein Quantenteilchen bekannter Masse für die maximalen Impulse hätte. Da
\begin{equation}
   W_{kin} = \frac{1}{2} \cdot m \cdot v^2 = \frac{1}{2} \cdot \frac{p_x^2}{m}
\end{equation}
gilt, kann man $p$ über Gleichung \ref{eq_ubr} ausdrücken. Dazu ersetzt man $\Delta x$ noch durch $L$ -- $L$ ist die Länge im eindimensionalen Raum, auf die man das Quantenteilchen beschränkt. Somit ergibt sich für die Lokalisationsenergie in einer Dimension ($x$-Dimension):
\begin{equation}
   W_{lokal} = W_{kin} = \frac{1}{2} \cdot \frac{h^2}{4 \cdot L^2} \cdot \frac{1}{m} = \frac{h^2}{8 \cdot m \cdot L^2}
\end{equation}







\section{Photon als Quantenteilchen}

Auch beim Photon\index{Photon} handelt es sich um ein Quantenteilchen. Man kann seine Masse bestimmen und sogar nachweisen: Bewegt sich ein Photon gegen das Schwerefeld der Erde, so wandelt es Energie in potentielle Energie um. Wenn es um $s$ steigt, muss es die Energie $W_{ab} = m \cdot g \cdot s$ umwandeln. Dadurch verändert sich seine Frequenz um $f = \frac{W_{ab}}{h}$. Außerdem ist dem Photon ein Impuls zuzuschreiben. Er ergibt sich aus der Formel $E = m \cdot c^2$ der Relativitätstheorie, indem man die Energie $h \cdot f$ in Masse umwandelt und diese mit der einzigen Geschwindigkeit, bei der Photonen existieren ($c$), multipliziert:
\begin{equation}
   p_{Photon} = m \cdot c = \frac{h \cdot f}{c^2} \cdot c = h \cdot \frac{f}{c} = \frac{h}{\lambda}
\end{equation}
Für Photonen gilt also auch die Formel der \textsc{DeBroglie}-Wellenlänge.


Es ergeben sich jedoch einige Unterschiede zum Elektron (welches bisher als das Beispielquantenteilchen verwendet wurde):
\begin{itemize}
   \item Photonen existieren nur bei $v = c$; Elektronen können auch in Ruhe existieren
   \item Elektronen gehören zur Materie
   \item Photonen werden nicht von $E$-Feldern beeinflusst
   \item Das \textsc{Pauli}-Prinzip gilt nicht bei Photonen
\end{itemize}

